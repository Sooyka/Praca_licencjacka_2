\documentclass[a4paper, 12pt]{article}
\usepackage{polski}
\usepackage{amssymb}
\usepackage[english,polish]{babel}
\usepackage[utf8]{inputenc}
\usepackage{mathtools}
\usepackage{amsthm}
\usepackage{amsfonts}
\usepackage{empheq}
\usepackage{xcolor}
\usepackage[pdftex,
            pdfauthor={Bartosz Sójka},
            pdftitle={Praca licencjacka},
            pdfsubject={Równoważność przez cięcia w przestrzeni dwuwymiarowej}]{hyperref}

%\setlength\parindent{0pt}

\newtheorem{observation}{Observation}
\newtheorem{definition}{Definition}
\newtheorem{theorem}{Twierdzenie}
\newtheorem{lemma}{Lemat}

\newcommand{\todo}[1]{\hfill \break \textbf{\Huge \textcolor{violet}{TO DO: #1} \hfill \break}\normalsize}
\newcommand{\content}[1]{\hfill \break \textbf{\large \textcolor{violet}{#1} \hfill \break}\normalsize}
\newcommand{\smalltodo}[1]{\textbf{\ \textcolor{violet}{To do}}}
\newcommand{\onedotsn}[1]{{#1}_1, \dots, {#1}_n}
\newcommand{\ndotsm}[3]{{#1}_{#2}, \dots, {#1}_{#3}}


\begin{document}
The paper will be rwritten in the ascending level of difficulty from to abstract
nonsense.
The motivatiopn was the question how to
Netherthe less is completly not rigurous
Punktem wyjścia była próba.



\begin{abstract}
W roku bla bla Bla Bla udowodnił że bla bla. Motywacją był prosty dowód ,że jest to niemożliwe dla cięć
kawałkami $C^2$. Praca będzie napisała z rosnącą trudnością.
\end{abstract}
\section{}
\content{\textit{rysunki okręgu i kwadratu}}
Mając do pocięcia okrąg i kwadrat spodziewamy się, że to sie nie uda, jako że nie ma w co wpasować
obłości jakie ma okrąg, pjawia się pytanie, czy nie da sie tak jakoś sprytnie powycinać, żeby te obłości
zniwelować. Kiedy sie nad tym zastanowimy, możemy spodziewać sie, że nie, pownieważ próba wycięcia wklęsłośi
w którą wpasuje się wypukłość tworzy kolejną wypukłość \\
\content{\textit{rysunek}} \\
Nie mniej jednak nie jest to precyzyjny argument i nie jesteśmy pewni, czy na pewno nie da isę tak powycinać
części z koła, zeby nie złożyć kwadratu.
Dlatego powiemy teraz tamten pomysł precyzyjniej. \\
\content{\textit{ewentualny rozdział o krzywiźnie (dlaczego ona, itp.) }} \\
Zauważmy, że całka z krzywizny nie zmienia się przy cięciach, sklejeniach i przesunięciach.
OK
Koło ma $\displaystyle\int_{\partial} \kappa = 2 \pi$, a kwadrat ma $\displaystyle\int  _{\partial} \kappa
= 0$. \\
Nie rozwiązuje to kwestii do końca, jako, że nie odpowiada na pytanie czy da się podzielić pierścień:
\content{\textit{rysunek pierścienia}}
w kwadrat. Całka w obu przypadkach to zero. Widzimy jednak, że w przypadku pierścienia ujemna wartość
wniesiona jest przez większą krzywiznę na mniejszej długości zaaś dotatnia przez mniejszą krzywizn na
wiekszej długości. Intuicja podpowiada, że pierscienia równiż nie da sie rozciąć, bo nie każdy punkt ma
jednakowe odpowiadające mu otoczenie. Zeby to uwzględnić wprowadzamy znakowaną miarę obłości.
\content{\textit{rozdział o znakowanej mierze obłości}}
bla bla
\section{poszukiwanie odwrotnego warunku}
\section{tu napiszę dowód, że wystarczy przenieść brzeg z zachowaniem skierowania}

Załóżmy, że dwie figury mają tę własność, że mogę pociąć brzeg jednej z nich tak, że można go przełożyć na
brzeg drugiej oraz nie mają szpiców. Pokażemy, że wtedy da się tak pociąć całą figurę. \\
1. Niech $\alpha$ to najmniejszy kąt jaki tworzą wektory styczne do końców krzywych po sklejeniu.
2. Wybierzmy $\varepsilon$ tak, by w każdej kuli o promieniu $\varepsilon$ przeciwobraz $\gamma^-{1}\gamma
\cup B$ był spójny oraz by $\gamma^{''}$ było stałego znaku po obu stronach punktu. Oraz by linia
poprowadzona z punktu rozcięcia pod kątem $\alpha/2$ przecinała jako prierwsze linię pasu złożoną z kul a
nie krzywą (czy tak się da?) (da się trzeba zobaczyć jak daleko krzywa oddala się od linii i zmiejszyć
$\varepsilon$ poniżej tego). wtedy części obraniczone tymi pasami i liniami po podzieleniu odpowiednio
drobno dają się przenieść nie tracąć pola powierzchi. I sprowadza sie to do problemu podziału wielokątów.

\content{jak mają równe pola i da sie przenieść brzegi, to da się całe figury}
SUper! \\
\content{jeśli jest jakiś niezmiennik, który nie rozróżnia dwóch krzywych gładkich, to nie rozróżni dla
obszarów}
więc jeśli dla jakiegoś niezmiennika zobaczymy że zawodzi on na krzywych gładkich, to zawodzi on na
figurach \\

\section{podejście drugie}


Definicja jest następująca
\end{document}
