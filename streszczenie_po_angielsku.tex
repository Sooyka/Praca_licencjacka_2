\documentclass[a4paper, 12pt, twosided]{article}
\synctex=1
\usepackage{polski}
\usepackage[T1]{fontenc}
\usepackage{amssymb}
\usepackage[polish, english]{babel}
\usepackage[utf8]{inputenc}
\usepackage{microtype}
\usepackage{mathtools}
\usepackage{amsthm}
\usepackage{amsfonts}
\usepackage{empheq}
\usepackage{xcolor}
\usepackage[pdftex,
            pdfauthor={Bartosz Sójka},
            pdftitle={Równoważność przez cięcia w przestrzeni dwuwymiarowej},
        %    pdfsubject={Równoważność przez cięcia w przestrzeni dwuwymiarowej}
            ]{hyperref}
\usepackage{siunitx}
\usepackage{geometry}
%\setlength\parindent{0pt}

%\topmargin = -1in

\geometry{a4paper, 
twoside,
%asymmetric,
top = 35mm,
bottom = 45mm,
inner = 35mm,
outer = 30mm}

%\linespread{1.24}

%top = 25mm,
%bottom = 30mm,
%inner = 30mm,
%outer = 25mm

\newtheorem{observation}{Observation}[subsection]
\newtheorem{definition}[observation]{Definicja}
\newtheorem{theorem}[observation]{Twierdzenie}
\newtheorem{lemma}[observation]{Lemat}

\newcommand{\todo}[1]{\hfill \break \textbf{\Huge \textcolor{violet}{TO DO: #1} \hfill \break}
\normalsize}
\newcommand{\content}[1]{\hfill \break \textbf{\large \textcolor{violet}{#1} \hfill \break}
\normalsize}
\newcommand{\smalltodo}[1]{\textbf{\ \textcolor{violet}{To do}}}
\newcommand{\smalltodoII}[1]{\hfill \break \textbf{\ \textcolor{violet}{To do: #1}}\hfill \break}
\newcommand{\onedotsn}[1]{{#1}_1, \dots, {#1}_n}
\newcommand{\ndotsm}[3]{{#1}_{#2}, \dots, {#1}_{#3}}
\newcommand{\rysunek}[1]{\hfill \break\\[16pt] \Huge \textbf{\textcolor{violet}{Brakujący rysunek 
\normalsize
#1}} \hfill
\break \\[16pt] \normalsize}
\newcommand{\colvect}[2]{\begin{pmatrix}#1 \\ #2\end{pmatrix}}

\title{Równoważność przez cięcia w przestrzeni dwuwymiarowej}
\author{Bartosz Sójka}
%\date{}

\begin{document}
\begin{abstract}
    W roku 1990 
    %węgierski matematyk 
    Miklós Laczkovich rozwiązał problem kwadratury koła Tarskiego -- 
    udowodnił
     on, że koło da się podzielić na skończoną liczbę części, z których można ułożyć kwadrat 
     \cite{Laczkovich1990}. Wynik ten został wzmocniony w pracy autorstwa 
     Łukasza Grabowskiego, Andrása Máthé'a i 
     Olega Pikhurko, w której pokazali oni, że części na które dzielone są figury mogą być zbiorami 
      Baira
     mierzalnymi w sensie Lebesgue'a~\cite{Grabowski}. 
     %, a przemieszczać wystarczy je poprzez 
     %translacje
     %~\cite{Grabowski}. 
%     \footnote{M. Laczkovich. Equidecomposability and discrepancy; a solution of
%Tarski’s circle-squaring problem. (404):77–117, 1990.}
     %Twierdzenie to może 
     Twierdzenia te mogą 
     wydawać się nieintuicyjne i istotnie, na przykład części na które zostało podzielone 
      koło w 
     %dowodzie 
     dowodach 
     były zbiorami 
     %niemierzalnymi
     niehomeomorficznymi z $D^2$,
      a 
     %sam dowód 
     same dowody
     %był
     były 
     %niekonstruktywny. 
     niekostruktywne.
     Co więcej, twierdzenia te nie 
     %jest ono
     są 
     prawdziwe, w przypadku
      gdy
     ograniczymy się do podziałów koła na zbiory, których brzegi są prostowalnymi krzywymi Jordana, 
     co pokazali Lester Dubins, Morris W. Hirsch i Jack Karush w~\cite{[DHK]}. 
      Punktem 
     wyjścia
     pracy jest przypadek powyższego zagadnienia, gdzie brzegi są krzywymi Jordana kawałkami 
     $C^\infty$.
     Można traktować je jako dobry model dla fizycznie realizowalnych cięć. 
     Użyte argumenty oraz sposób rozumowania są w swojej naturze podobne do 
     %przypominają 
     tych z~\cite{[DHK]}. Dalej będzie rozwinięta
      i
     omówiona teoria klasyfikacji figur w przestrzeni dwuwymiarowej ze względu na ich równoważność 
     przez
     cięcia.
 \end{abstract}
\begin{abstract}
In the year 1990 Miklós Laczkovich solved Tarski's problem of squaring the circle -- he proved, that 
a circle can be 
%decomposed 
divided into finitely many pieces from which one can compose a square\cite{Laczkovich1990}. 
This result was strengten in the paper written by Łukasz Grabowski, András Máthé and  
     Oleg Pikhurko where they showed that the pieces into which figures are divided can be 
     Bair sets, mesurable in the Lebesgue sense~\cite{Grabowski}. These theorems can seem 
     unintuitive and indeed, for example, the pieces in which the circle was divided in the proofs 
     were sets that were non-homeomorphic with $D^2$ and the proofs themselfs was unconstructive 
     %ones. 
     . What is more, these theorems are not true in the case,  where we restrict ourselfs to 
     divisions of the circle into pieces which boundaries are Jordan curves what was shown by Lester 
     Dubins, Morris W. Hirsch and Jack Karush in~\cite{[DHK]}. The starting point of the paper is 
     the case of above problem, where boundries are Jordan curves, piecewise $C^\infty$. They can be 
     treated as a good model for physically-realisable cuts. Argumets used and reasoning are 
     simmilar 
     in nature to those in~\cite{[DHK]}. Next there will be futher developed and discussed the 
     theory of classification of figures in two dimentional space with respect to their equivalence 
     by cutting.
     
\end{abstract}
Scissor-congruence in a two dimentional space
\nocite{*}
\bibliography{praca_bibliografia}{}
\bibliographystyle{plain}
\end{document}
