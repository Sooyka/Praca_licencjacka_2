\documentclass[a4paper, 12pt]{article}
\usepackage{polski}
\usepackage{amssymb}
\usepackage[polish, english]{babel}
\usepackage[utf8]{inputenc}
\usepackage{mathtools}
\usepackage{amsthm}
\usepackage{amsfonts}
\usepackage{empheq}
\usepackage{xcolor}
\usepackage[pdftex,
            pdfauthor={Bartosz Sójka},
            pdftitle={Równoważność przez cięcia w przestrzeni dwuwymiarowej},
        %    pdfsubject={Równoważność przez cięcia w przestrzeni dwuwymiarowej}
            ]{hyperref}

%\setlength\parindent{0pt}

\newtheorem{observation}{Observation}
\newtheorem{definition}{Definition}
\newtheorem{theorem}{Twierdzenie}
\newtheorem{lemma}{Lemat}

\newcommand{\todo}[1]{\hfill \break \textbf{\Huge \textcolor{violet}{TO DO: #1} \hfill \break}\normalsize}
\newcommand{\content}[1]{\hfill \break \textbf{\large \textcolor{violet}{#1} \hfill \break}\normalsize}
\newcommand{\smalltodo}[1]{\textbf{\ \textcolor{violet}{To do}}}
\newcommand{\onedotsn}[1]{{#1}_1, \dots, {#1}_n}
\newcommand{\ndotsm}[3]{{#1}_{#2}, \dots, {#1}_{#3}}

\title{Równoważność przez cięcia w przestrzeni dwuwymiarowej}
\author{Bartosz Sójka}
%\date{}

\begin{document}
\thispagestyle{empty}
\begin{center}
\textbf{\large Uniwersytet Wrocławski\\
Wydział Matematyki i Informatyki\\
Instytut Matematyczny}\\
\textit{\large specjalność teoretyczna}\\
\vspace{4cm}
\textbf{\textit{\large Bartosz Sójka}\\
\vspace{0.5cm}
{\Large Równoważność przez cięcia w przestrzeni dwuwymiarowej}}\\
\end{center}
\vspace{3cm}
{\large \hspace*{6.5cm}Praca magisterska\\
\hspace*{6.5cm}napisana pod kierunkiem\\
\hspace*{6.5cm}prof. dr. hab. Jana Dymary }\\
\vfill
\begin{center}
{\large Wrocław Rok 2019}\\
\end{center}
\newpage
\null
\thispagestyle{empty}
\newpage
\tableofcontents

\begin{abstract}
    W roku 1990 wegierski matematyk Miklós Laczkovich rozwiązał problem kwadratury koła Tarskiego - udowodnił
     on, że koło da się podzielić na skończoną liczbę części, z których można ułożyć kwadrat. Twierdzenie
     to może wydawać się nieintuicyjne i istotnie, części na które zostało podzielone koło w dowodzie były
     zbiorami niemieżalnymi, a sam dowód był niekonstruktywny. Co więcej, nie jest ono prawdziwe, gdy
     ograniczymy się do podziałów koła na zbiory, których brzegi są krzywymi Jordana. Punktem wyjścia
     pracy jest przypadek powyższego zagadnienia, gdzie brzegi są krzywymi Jordana kawałkami $C^\infty$.
     Można traktować je jako dobry model dla fizycznie realizowalnych cięć. Dalej będzie rozwinięta i
     omówiona teoria klasyfikacji figur w przestrzeni dwuwymiarowej ze względu na ich równoważność przez
     cięcia.
 \end{abstract}

 \section{Koło i kwadrat}
Praca będzie miała stopniowo rosnący poziom formalności. Początek jest gawędą o moich rozmyślaniach i
motywacjach skąd wzięły się przedstawione problemy. Mniej więcej między  drugim a trzecim rozdziałem
całość nabiera formalizmu. Przez całą pracę rozważane krzywe są krzywymi kawałkami $C^\infty$, gdzie kolejne
$C^\infty$ fragmenty krzywych nigdy nie tworzą kąta $0stopni$ (tutaj formalniej). Będą one dalej nazywane
po prostu krzywymi. \\
łuk - spójny fragment okręgu \\[16pt]

Intuicja podwpowiada, że biorąc koło i tnąc je na skończenie wiele części nie dostaniemy takich, z których
da się ułożyć kwadrat. Co stoi za tą intuicją? Widać, że ewidentnym problemem jest brzeg koła, który jest
zaokraglony. Kwadrat żadnych zaokrągleń nie ma. Czy jednak aby na pewno to przeszkadza? Z pewnością, gdy
tniemy koło i dostaniemy kawałek, którego fragment brzegu jest fragmentem okręgu, to nie da się za pomocą
translacji i obrotów tego fragmentu odwzorować na fragment brzegu kwadratu. Jedno jest obłe, drugie jest
płaskie. Pojawia się jednak wątpliwość, czy jeśli potnie się koło na masę kawałków i część z tych kawałków
będzie miała odcinki jako krawędzie, które zostaną odwzorowane na brzeg kwadratu, to na przykłąd nie da się
tak wyciąćtych części, by "upchnąć" gdzieś również obłość okręgu. Przeciwko takiej możliwości świadczy
następujące rozumawanie. \\
Kiedy tnę figurę krzywą tak, żeby wprowadzić wklesłość, pojawia mi się również komplementarna do niej obłość.
\\
rysunek
\\
Można spodziewać się więc, że próbując wyprodukować dla okręgu nadmiarową wklęsłość, starania te zawiodą,
gdyż za każdym razem powstanie tyle samo obłości. Tylko co to w tym wypadku znaczy "tyle samo"?
Nie wiedząc tego dalej pozostaje pewna wątpliwość, czy nie da się pociąć tak, by obłość zniknęła. widać, że
przy cięciach generujących wklęsłość obłość się pojawia, ale może pojawia się jej "mniej". Powinniśmy w takim
razie wprowadzić aparat pojęciowy pozwalający nam mierzyć "ilość" wklęsłości i obłości, tak by zobaczyć, czy
przy jakimkolwiek cięciu, ta się wyprodukować jedno, nie produkujac jednakowej "ilości" drugiego. \\[4pt]
Przystąpimy do szukania satysfakcjonującej definicji obłości figury, która pozwoli nam rozróżnić
koło i kwadrat i sformalizować nasze intuicje. Obłość ma oddawać fakt, że coś jest zaokrąglone, więc dla bardziej
zaokrąglonych obiektów chcielibyśmy, żeby była większa, a dla mniej, żeby była mniejsza. Ponadto o obłości
figury ma świadczyć sam kształt brzegu danej figury. Zdefiniujmy zatem obłość dla krzywych. Powiemy, że
obłością figury z definicji jest tak zdefiniowana obłość jej brzegu. \\
Niech dowolnego odcinka jego obłość wynosi zero. Niech dla dowolnego okręgu jego obłość wynosi $2\pi$.
Niech dla spójnego fragmentu okregu jego obłość do obłości całego okręgu ma się tak jak jego długość do
długości całego okregu. Dzięki temu dla rodziny krzywych o tej samej długości $l$ - odcinka i fragmentów
okręgów wszystkich możliwych promieni ($[l/2\pi, \infty)$ faktycznie obłość jest tym większa im bardziej
zakrzywiona jest krzywa. \\
rysunek \\
Zdefiniujmy teraz obłość dla krzywych $C^2$. \\
rysunek - podział krzywej, okręgi \\
Definiujemy ją jako granicę przybliżeń obłości. \\
Przybliżamy obłość dzieląc krzywą na fragmenty i opisując okręgi tak jak na rysunku. Przybliżona obłość
fragmentu, to obłość odpowiadającego fragmentu okręgu. \\
Chcemy jednak rozróżniać "wklęsłość" od "wypukłości". Dlatego wybieramy stronę krzywej, która ma być
"wnętrzem" i sumujemy przyczynki z odpowiednimi znakami. \\

Granicą promieni okregów z kolejnyh przybliżeń w pojedynczym punkcie jest promień krzywej w danym punkcie.
Obłość dla łuku wyraża się jako długość przez promień. Skąd obłość krzywej jest całką po krzywej z
odwrotności promienia ze znakiem. Jest to więc całka po krzywej z jej krzywizny ze znakiem. \\
% Teraz możemy wyrażać obłość liczbowo w upożadkowanym ciele liczb rzezywistych.
Zauważmy teraz, że cięcie figury nie zmienia jej obłości. Istotnie każda krzywa cięcia generuje dwie krzywe
brzegów fragmentów, które są identyczne z wnętrzmi po przeciwnych stronach. \\
Zatem całki ze znakowanej krzywizny (obłość) po nich są identyczne co do wartości i przeciwne co do znaku.
Zatem ich suma wynosi zero. \\
Stąd po poccięciu figura ma identyczną obłość, co przed. \\
Przesuwanie fragmentów i obracanie oczywiście nie zmienia obłości. Równiż klejenie fragmentóœ ze sobą nie
może jej zmienić, gdyż klejone fragmenty muszą być przystające i mieć wnętrze po przeciwnch stronach. Suma
ich obłości musi wynosić zatem zero. \\


\section{Pierścień i kwadrat}

\section{Twierdzenie odwrotne}

\subsection{Równoważność z problemem na brzegach}

\subsection{Kiedy ZMO zawodzi}

\section{Wielka grupa abelowa}

\section{Równoważność elementu w wielkiej grupie abelowej z klasą figur równoważnych ze sobą przez cięcia}

\section{Dyskusja, że to niewiele daje}

\section{a co, jeśli, krzywe są zamknięte i (nie kawałkami!) głądkie?}




\end{document}
