\documentclass[a4paper, 12pt]{article}
\usepackage{polski}
\usepackage{amssymb}
\usepackage[english,polish]{babel}
\usepackage[utf8]{inputenc}
\usepackage{mathtools}
\usepackage{amsthm}
\usepackage{amsfonts}
\usepackage{empheq}
\usepackage{xcolor}
\usepackage[pdftex,
            pdfauthor={Bartosz Sójka},
            pdftitle={Równoważność przez cięcia w przestrzeni dwuwymiarowej},
        %    pdfsubject={Równoważność przez cięcia w przestrzeni dwuwymiarowej}
            ]{hyperref}

%\setlength\parindent{0pt}

\newtheorem{observation}{Observation}
\newtheorem{definition}{Definition}
\newtheorem{theorem}{Twierdzenie}
\newtheorem{lemma}{Lemat}

\newcommand{\todo}[1]{\hfill \break \textbf{\Huge \textcolor{violet}{TO DO: #1} \hfill \break}\normalsize}
\newcommand{\content}[1]{\hfill \break \textbf{\large \textcolor{violet}{#1} \hfill \break}\normalsize}
\newcommand{\smalltodo}[1]{\textbf{\ \textcolor{violet}{To do}}}
\newcommand{\onedotsn}[1]{{#1}_1, \dots, {#1}_n}
\newcommand{\ndotsm}[3]{{#1}_{#2}, \dots, {#1}_{#3}}

\title{Równoważność przez cięcia w przestrzeni dwuwymiarowej}
\author{Bartosz Sójka}
%\date{}

\begin{document}
\thispagestyle{empty}
\begin{center}
\textbf{\large Uniwersytet Wrocławski\\
Wydział Matematyki i Informatyki\\
Instytut Matematyczny}\\
\textit{\large specjalność teoretyczna}\\
\vspace{4cm}
\textbf{\textit{\large Bartosz Sójka}\\
\vspace{0.5cm}
{\Large Równoważność przez cięcia w przestrzeni dwuwymiarowej}}\\
\end{center}
\vspace{3cm}
{\large \hspace*{6.5cm}Praca magisterska\\
\hspace*{6.5cm}napisana pod kierunkiem\\
\hspace*{6.5cm}prof. dr. hab. Jana Dymary }\\
\vfill
\begin{center}
{\large Wrocław Rok 2019}\\
\end{center}
\newpage
\null
\thispagestyle{empty}
\newpage
\tableofcontents

\begin{abstract}
    W roku 1990 wegierski matematyk Miklós Laczkovich rozwiązał problem kwadratury koła Tarskiego - udowodnił
     on, że koło da się podzielić na skończoną liczbę części, z których można ułożyć kwadrat. Twierdzenie
     to może wydawać się nieintuicyjne i istotnie, części na które zostało podzielone koło w dowodzie były
     zbiorami niemieżalnymi, a sam dowód był niekonstruktywny. Co więcej, nie jest ono prawdziwe, gdy
     ograniczymy się do podziałów koła na zbiory, których brzegi są krzywymi Jordana. Punktem wyjścia
     pracy...
 \end{abstract}



\end{document}
